%---------------------------------------------------------------------------
\documentclass[fontsize=12pt,paper=a4,draft=off,titlepage=off]{scrartcl}
%---------------------------------------------------------------------------
\usepackage[T1]{fontenc}
\usepackage[utf8]{inputenc}
\usepackage{CJKutf8}
\usepackage{url}
\usepackage[german]{babel}
\usepackage{graphicx}
\usepackage{palatino}
\usepackage{inputenc}
\usepackage{fancyhdr}
\usepackage{booktabs}
\usepackage{dcolumn}
\usepackage{longtable}
\usepackage{epstopdf}
\usepackage{pgf-pie} % /home/nils/git/abschlussprojekt/documentation/pgf-pie
\usepackage{enumitem}
\setlist[enumerate]{label*=\arabic*.}
\setlength\headsep{2cm}
\lhead{}
\chead{}
\rhead{\includegraphics[bb=0 0 667 57, scale=0.5]{/home/freundt/business/GA_Logo.jpeg}}

\newcolumntype{d}{D{.}{.}{4}}

\title{Projektantrag\\ \textbf{Ausbildung zum Fachinformatiker f. Anwendungsentwicklung}}
\author{Nils Diefenbach}
\date{Abschlussprüfung Winter 2017\\ \small{Version vom \today}}

%---------------------------------------------------------------------------
\begin{document}
\pagestyle{fancy}
%\maketitle

\begin{titlepage}
\centering
{\scshape\LARGE Projektantrag\par}
\vspace{1cm}
{\scshape\Large Ausbildung zum Fachinformatiker f. Anwendungsentwicklung\par}
\vspace{1.5cm}
{\huge\bfseries Abschlussprüfung Winter 2017\par\\IHK Berlin\par}
\vspace{2cm}
Auszubildender:\par
{\Large\itshape Nils Diefenbach\par}
Ident-Nummer:\par
{\Large\itshape 3590244\par}
\vfill
Ausbilder:\par
Sebastian Freundt\par \textsc{GA Financial Solutions GmbH}
\vfill
{\large \today\par}
\end{titlepage}

\clearpage

\tableofcontents{}

\clearpage

\section{Projektbezeichnung}
Verbindung von Tabellen auf Basis eines Fuzzy-Vergleichsalgorithmus.\par

\subsection{Aufgabenstellung}
Für die GA Financial Solutions GmbH soll die Assoziierung von Namen diverser
Wertpapiere von unterschiedlichen Datenanbietern effizienter und fehlerfreier
gestaltet werden.\par
Hierzu wird ein Tool entwickelt, welches sowohl über Kommandozeile und Unix
Pipes nutzbar ist, als auch in Python Code als Bibliothek einbindbar ist und
den Vergleich mit Hilfe einer Fuzzy-String-Suche bewerkstelligt.\par

\subsection{Ist-Zustand}
Die GA Financial Solutions handelt Wertpapiere aller Art auf eigene
Rechnung.  Dabei ist es marktüblich, dass Referenzdaten, Preisdaten
und Ausführungen von unterschiedlichen Dienstleistern bezogen werden.
So stehen ca. 223 Mio. handelbaren Wertpapieren rund 500000 mögliche
Herausgeber an ca. 1200 Börsen weltweit gegenüber.  Jede einzelne
Börse bzw. jeder einzelne Makler im Markt unterstützt jedoch nur
einen Bruchteil der Papiere.\par

In Ermangelung eines weltweiten Standards, der jedem Marktteilnehmer in
jeder Handelsphase gerecht wird, werden unterschiedliche Symbologien
benutzt, deren kleinster gemeinsamer Nenner stets die
Klartextbezeichnung des begebenen Papiers zusammen mit der
Klartextbezeichnung des Herausgebers ist.  Erschwert wird die
Problematik überdies noch durch unterschiedliche Transliterationsverfahren,
so führt z.B. die NASDAQ die Firma "`Panasonic Corp."',
die GLEIS-Datenbank\footnote{Global Legal Entity Identifier System} jedoch unter
\begin{CJK}{UTF8}{min}
"パナソニック株式会社"
\end{CJK}%
, was transskribiert wiederum "`Panasonikku Kabushiki-gaisha"' ergibt. \par

Bei der GA Financial Solutions werden Zuordnungen zwischen Symbologien
aktuell ad-hoc mit Heuristiken und Handarbeit bewerkstelligt.  Der
Zeitaufwand ist entsprechend hoch, die Fehlerrate ebenfalls.  Es
verbietet sich geradezu, die bisherigen Heuristiken systematisch zur
Zuordnung aller Herausgeber aller Wertpapiere zu benutzen.\par


\clearpage

\section{Soll-Konzept}

\subsection{Was soll am Ende des Projektes erreicht sein?}
Mithilfe des Tools soll es möglich sein, die Entitäten aus zwei gegebenen,
beliebig langen Listen anhand einer Metrik, Winkels oder Scores
so vergleichen zu können, dass die in den Listen enthaltenen Varianten
miteinander assoziierbar werden.\par

Hierbei liegt das Hauptaugenmerk auf der Fehlerreduktion (Reduzierung von so genannten "`False Negatives"') und
performanteren Bearbeitung dieser Listen.\par

Das Programm wird mit Hilfe von Unit-Tests geprüft und lässt sich
sowohl per Unix Pipe in unseren Toolchains einsetzen, als auch per Bibliothek in
Python nutzen. Für die Kommandozeilenvariante müssen die GNU Coding
Conventions und die POSIX Utilities Guidelines eingehalten werden.\par

\subsection{Welche Anforderungen müssen erfüllt sein?}
\begin{itemize}
\item Ein Command Line Interface muss vorhanden sein
\item Das CLI muss Daten via Pipe annehmen und verarbeiten können
\item Das Tool soll Teil eines Python Pakets sein, welches mit Python Version
3.x nutzbar ist.
\item Die Ergebnisse des Tools und des Pakets müssen mit geeigneten Tests überprüft werden
\item Die Leistungsfähigkeit des Tools soll in einem geeignetem Benchmark unter Beweis gestellt werden
\end{itemize}

\clearpage

\section{Projektstrukturplan}
\subsection{Was ist zur Erfüllung der Zielsetzung erforderlich}
Das Projekt wird nach dem Wasserfall-Prinzip entwickelt werden. Hierbei werden
Meilensteine zuvor festgelegt, welche mit Hilfe von Tests dokumentiert und nach
erreichen mit dem Fachbereich besprochen werden.
Die Entwicklung und Implementierung ist testgetrieben und wird mit Hilfe von
Unit-Tests und der NOSE-Testsuite durchgeführt.\par
Eine Versionierung des Projekts findet über ein Git Repository statt.

\subsection{Aufgabenliste}
\begin{enumerate}
\item Analysephase
    \begin{enumerate}
    \item Durchführung einer Ist-Analyse
    \item Durchführung einer Wirtschaftlichkeitsanalyse und Amortisationsrechnung
    \item Ermittlung von Use-Cases
    \item Erstellung eines Lastenheftes
    \end{enumerate}
\item{Entwurfsphase}
    \begin{enumerate}
    \item Komponentendiagramm
    \item Schnittstellenentwurf für Python
    \item Pflichtenheft
    \item Entwurf der Tests anhand des Pflichtenhefts
    \item Entwurf eines Benchmark-Tests
    \end{enumerate}
\item{Implementationsphase}
    \begin{enumerate}
    \item Implementierung des C Codes
    \item Erstellung des Python Wrappers
    \end{enumerate}
\item{Abnahme}
    \begin{enumerate}
    \item Eingliederung in Betriebsablauf
    \item Finales Benchmarking
    \end{enumerate}
\item{Dokumentationsphase}
    \begin{enumerate}
    \item Erstellung Projektdokumentation
    \item Erstellung Entwicklerdokumentation
    \end{enumerate}
\end{enumerate}

\subsection{Grafische und tabellarische Darstellung}

\begin{tikzpicture}
    \pie [rotate = 180]
    {9/Analyse, 31/Entwurf, 43/Implementierung, 4/Abnahme, 13/Dokumentation}
\end{tikzpicture}

\begin{table}[!htp]
    \centering
    \label{Übersicht Phasendauer}
    \begin{tabular}{ll}
        Phase           & Dauer in Stunden \\
        \hline
        Analyse         & 6  \\
        Entwurf         & 22 \\
        Implementierung & 30 \\
        Abnahme         & 3  \\
        Dokumentation   & 9  \\
        \hline
        Summe           & 70 \\
    \end{tabular}
\end{table}


\clearpage
\section{Projektphasen mit Zeitplanung in Stunden}
\begin{table}[!htp]
    \centering
    \label{Übersicht Phasendauer}
    \begin{tabular}{ll}
        \textbf{Analysephase} & \textbf{6h} \\
        \hline
            Ist-Analyse & 2h \\
            Wirtschaftlichkeitsanalyse und Amortisationsrechnung & 1h \\
            Erstellung eines Lastenheftes & 3h \\

        \textbf{Entwurfsphase} & \textbf{22h} \\
        \hline
            Komponentendiagramm erstellen & 1h \\
            Schnittstellenentwurf für C und Python & 3h \\
            Aktivitätsdiagramm zur Nutzung via Pipe & 1h \\
            Pflichtenheft & 5h \\
            Entwurf der Tests anhand des Pflichtenhefts (User Stories) & 7h \\
            Entwurf eines Benchmark-Tests & 5h \\

        \textbf{Implementationsphase} & \textbf{30h} \\
        \hline
            Implementierung des C Codes mit Tests & 21h \\
            - Implementierung eines Suffix-Trie & \textit{8h} \\
            - Implementierung Fuzzy-Such-Algorithmus & \textit{8h} \\
            - Implementierung Scoring-System & \textit{5h} \\
            Erstellung des Python Wrappers mit Tests & 5h \\
            - C Code über Cython Einbinden & \textit{2h} \\
            - Bibliothek erstellen und optimieren & \textit{3h} \\

        \textbf{Abnahme und Deployment} & \textbf{3h} \\
        \hline
            Eingliederung in Betriebsablauf & 2h \\
            - Eingliederung in Cronjobs und automatiesiert Scripts & \textit{1h} \\
            - Deployment in global-installiertes Toolset & \textit{1h} \\
            Finales Benchmarking & 1h \\

        \textbf{Dokumentationsphase} & \textbf{9h} \\
        \hline
            Erstellung Projektdokumentation & 7h \\
            Erstellung Entwicklerdokumentation & 2h \\
    \end{tabular}
\end{table}
%---------------------------------------------------------------------------
\end{document}
%---------------------------------------------------------------------------

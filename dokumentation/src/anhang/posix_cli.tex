\say{
12.2 Utility Syntax Guidelines \\

The following guidelines are established for the naming of utilities and for the
specification of options, option-arguments, and operands. The getopt() function
in the System Interfaces volume of POSIX.1-2008 assists utilities in handling
options and operands that conform to these guidelines.\par

Operands and option-arguments can contain characters not specified in the portable
character set.\par

The guidelines are intended to provide guidance to the authors of future utilities,
such as those written specific to a local system or that are components of a
larger application. Some of the standard utilities do not conform to all of these
guidelines; in those cases, the OPTIONS sections describe the deviations.\par

\begin{itemize}
\item Guideline 1: Utility names should be between two and nine characters, inclusive.
\item Guideline 2: Utility names should include lowercase letters (the lower character classification) and digits only from the portable character set.
\item Guideline 3: Each option name should be a single alphanumeric character (the alnum character classification) from the portable character set. The -W (capital-W) option shall be reserved for vendor options. Multi-digit options should not be allowed.
\item Guideline 4: All options should be preceded by the '-' delimiter character.
\item Guideline 5: One or more options without option-arguments, followed by at most one option that takes an option-argument, should be accepted when grouped behind one '-' delimiter.
\item Guideline 6: Each option and option-argument should be a separate argument, except as noted in Utility Argument Syntax, item (2).
\item Guideline 7: Option-arguments should not be optional.
\item Guideline 8: When multiple option-arguments are specified to follow a single option, they should be presented as a single argument, using <comma> characters within that argument or <blank> characters within that argument to separate them.
\item Guideline 9: All options should precede operands on the command line.
\item Guideline 10: The first -- argument that is not an option-argument should be accepted as a delimiter indicating the end of options. Any following arguments should be treated as operands, even if they begin with the '-' character.
\item Guideline 11: The order of different options relative to one another should not matter, unless the options are documented as mutually-exclusive and such an option is documented to override any incompatible options preceding it. If an option that has option-arguments is repeated, the option and option-argument combinations should be interpreted in the order specified on the command line.
\item Guideline 12: The order of operands may matter and position-related interpretations should be determined on a utility-specific basis.
\item Guideline 13: For utilities that use operands to represent files to be opened for either reading or writing, the '-' operand should be used to mean only standard input (or standard output when it is clear from context that an output file is being specified) or a file named -.
\item Guideline 14: If an argument can be identified according to Guidelines 3 through 10 as an option, or as a group of options without option-arguments behind one '-' delimiter, then it should be treated as such.

[...]
}

src http://pubs.opengroup.org/onlinepubs/9699919799/basedefs/V1_chap12.html
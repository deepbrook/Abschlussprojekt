\say{4.7 Standards for Command Line Interfaces\\

It is a good idea to follow the POSIX guidelines for the command-line options of
a program. The easiest way to do this is to use getopt to parse them. Note that
the GNU version of getopt will normally permit options anywhere among the arguments
unless the special argument ‘--’ is used. This is not what POSIX specifies; it
is a GNU extension.\par

Please define long-named options that are equivalent to the single-letter
Unix-style options. We hope to make GNU more user friendly this way. This is
easy to do with the GNU function getopt_long.\par

One of the advantages of long-named options is that they can be consistent from
program to program. For example, users should be able to expect the “verbose”
option of any GNU program which has one, to be spelled precisely ‘--verbose’. To
achieve this uniformity, look at the table of common long-option names when you
choose the option names for your program (see Option Table).\par

It is usually a good idea for file names given as ordinary arguments to be input
files only; any output files would be specified using options (preferably ‘-o’
or ‘--output’). Even if you allow an output file name as an ordinary argument for
compatibility, try to provide an option as another way to specify it. This will
lead to more consistency among GNU utilities, and fewer idiosyncrasies for users
to remember.\par

All programs should support two standard options: ‘--version’ and ‘--help’. CGI
programs should accept these as command-line options, and also if given as the
PATH_INFO; for instance, visiting ‘http://example.org/p.cgi/--help’ in a browser
should output the same information as invoking ‘p.cgi --help’ from the command
line.\par

[...]

4.7.1 --version \\

The standard --version option should direct the program to print information
about its name, version, origin and legal status, all on standard output, and
then exit successfully. Other options and arguments should be ignored once this
is seen, and the program should not perform its normal function.\par

The first line is meant to be easy for a program to parse; the version number
proper starts after the last space. In addition, it contains the canonical name
for this program, in this format:\par

GNU Emacs 19.30 \\

The program’s name should be a constant string; don’t compute it from argv[0].
The idea is to state the standard or canonical name for the program, not its file
name. There are other ways to find out the precise file name where a command is
found in PATH.\par

[...]

4.7.2 --help \\

The standard --help option should output brief documentation for how to invoke
the program, on standard output, then exit successfully. Other options and
arguments should be ignored once this is seen, and the program should not perform
its normal function.\par

Near the end of the ‘--help’ option’s output, please place lines giving the email
address for bug reports, the package’s home page (normally ‘http://www.gnu.org/software/pkg’,
and the general page for help using GNU programs. The format should be like
this: \par

Report bugs to: mailing-address \\
pkg home page: <http://www.gnu.org/software/pkg/> \\
General help using GNU software: <http://www.gnu.org/gethelp/> \\

It is ok to mention other appropriate mailing lists and web pages.\par

}

src https://www.gnu.org/prep/standards/standards.html

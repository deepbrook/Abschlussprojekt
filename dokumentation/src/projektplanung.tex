\section{Projektplanung}
\label{section:projektplanung}
\subsection{Projektphasen}
Dem Autor standen 70 Arbeitsstunden zur Umsetzung des Projektes zur Verfügung.
Vor Beginn des Projekts wurden diese auf die, für die Softwareentwicklung
typischen, Phasen verteilt. Eine ausführlichere Aufteilung der Stunden kann der Tabelle
''Detaillierte Zeitplanung'' (\ref{table:zeitplanung}) entnommen werden.



\subsection{Abweichungen vom Projektantrag}
Ursprünglich wurde im Projektantrag eine Einbindung des Algorithmus in
Python genannt. Diese wurde, nach gründlicherer Recherche und in Absprache mit der
Abteilung Datenverarbeitung, jedoch auf einen späteren Zeitpunkt verlegt und nicht
Teil des Pflichtenhefts. Grund hierfür ist, dass die C-Implementierung eine höhere
Priorität hat als die Python Bibilothek, der Einbindungsprozess letzterer jedoch komplexer
ist als zuvor angenommen.
Deshalb wurde beschlossen die 7 Stunden, welche der Python-Implementierung
zugewiesen wurden, stattdessen der Optimierung des C-Programms zuzuweisen.

\subsection{Ressourcenplanung}
In der Ressourcenplanung finden sich sämtliche Einheiten aufgelistet, die für
 die Realisierung des Projektes benötigt wurden. Hierbei ist zu erwähnen, dass mit "`Einheiten"´ Hard- und Softwareressourcen, sowie personelle Ressourcen gemeint sind.
 Die Auflistung der Ressourcen kann im Abschnitt \ref{auszug:ressourcen} des Anhangs eingesehen werden.

\subsection{Entwicklungsprozess}
Für die Umsetzung des Projekts wurde vom Autor ein agiler Entwicklungsprozess in
Form von Kanban ausgewählt. Ausschlaggebend hierfür war die Tatsache, dass es bei Kanban
keine festen Phasenlängen gibt und Prioritäten sofort angepasst werden, sobald
neue Informationen (z.B. Abschluss von Arbeitspaketen, Verzögerungen im Ablauf) hinzukommen.\par

Das Kanbanboard wurde hierbei in fünf Bahnen aufgeteilt, welche repräsentativ für
die fünf Stufen, in welche der Autor die Implementierungsphase unterteilt hat, stehen:
\begin{itemize}
    \item \textbf{Offen} -- Arbeitspakete in dieser Spalte befinden sich noch
    nicht in der Entwicklung, und warten darauf bearbeitet zu werden.
    \item \textbf{Testentwicklung} -- Es werden Tests geschrieben, welche alle
    Anforderungen an den erforderlichen Code zur Fertigstellung des Arbeitspakets
    überprüfen.
    \item \textbf{Implementierung} -- Das Arbeitspaket wird bearbeitet und der dazugehörige Code erstellt. Hierbei wird Code geschrieben, der den Anforderungen
    des zuvor geschriebenen Tests gerecht wird.
    \item \textbf{Verifizierung} -- Das Arbeitspaket wurde fertiggestellt und
    muss nun von der Abteilung Datenverarbeitung abgesegnet werden.
    \item \textbf{Erledigt} -- Arbeitspakete in dieser Spalte sind fertiggestellt,
    vom Auftraggeber überprüft und unterschrieben worden.
\end{itemize}

Das Augenmerk liegt hierbei vor allem auf der Spalte "`Verifizierung"´.
Dieser bereichert den Entwicklungsprozess um einen formellen, fest vorgesehenen
Zeitpunkt zur Feedback-Kommunikation. So können frühzeitig, möglicherweise notwendige Anpassungen geäußert und unter Umständen noch umgesetzt werden. Wichtig hierbei ist, dass die Überprüfung nicht im Rahmen eines Meetings geschieht, da sonst der Entwicklungsprozess gestört wird. Ein Feedback zu den einzelnen Arbeitspaketen wird ausschließlich über das Kanbanboard kommuniziert.\par


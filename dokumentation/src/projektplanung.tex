\section{Projektplanung}
\label{section:projektplanung}
\subsection{Projektphasen}
Dem Autor standen 70 Arbeitsstunden zur Umsetzung des Projektes zur Verfügung.
Vor Beginn des Projekts wurden diese auf die, für die Softwareentwicklung
typischen, Phasen verteilt. Eine grobe Übersicht ist in der unteren Tabelle zu
entnehmen. Eine ausführlichere Aufteilung der Stunden kann der Tabelle
detaillierte Zeitplanung (\ref{tabelle:zeitplanung}) entnommen werden.



\subsection{Abweichungen vom Projektantrag}
Ursprünglich wurde im Projektantrag eine Implementierung des Algorithmus in
Python genannt. Diese wurde, nach gründlicherer Recherche und in Absprache mit der
Abteilung Datenverarbeitung jedoch auf einen späteren Zeitpunkt verlegt und nicht
Teil des Pflichtenhefts. Grund hierfür ist, dass die C Implementierung höhere
Priorität als die Python Bibilothek hat, der Einbindungsprozess letzterer jedoch komplexer
ist als zuvor angenommen.
Deshalb wurde beschlossen, die 7 Stunden, welche der Python Implementierung
zugewiesen worden waren, stattdessen der Optimierung des C Programms zu zuweisen.

\subsection{Ressourcenplanung}
In der Ressourcenplanung finden sich sämtliche Einheiten aufgelistet, welche für
 die Realisierung des Projektes benötigt wurden. Hierbei ist zu erwähnen dass
 hiermit Hard- und Softwareressourcen, sowie personelle Ressourcen gemeint sind.
 Die Auflistung der Ressourcen kann im Abschnitt \ref{auszug:ressourcen} des Anhangs eingesehen werden.

\subsection{Entwicklungsprozess}
Für die Umsetzung des Projekts wurde vom Autor ein agiler Entwicklungsprozess in
Form von Kanban ausgewählt. Ausschlaggebend waren hierfür die Tatsache, dass es bei Kanban
keine festen Phasenlängen gibt und Prioritäten sofort angepasst werden, wenn
neue Informationen(z.B. Abschluss von Arbeitspaketen, Verzögerungen im Ablauf) hinzukommen.\par

Das Kanbanboard wurde hierbei in 5 Bahnen aufgeteilt, welche repräsentativ für
die 5 Stufen, in welche der Autor die Entwicklungsphase unterteilt hat, stehen:
\begin{itemize}
    \item \textbf{Offen} -- Arbeitspakete in dieser Spalte befinden sich noch
    nicht in der Entwicklung, und warten darauf begonnen zu werden.
    \item \textbf{Testentwicklung} -- Es werden Tests geschrieben, welche alle
    Anforderungen an den erforderlichen Code zur Fertigstellung des Arbeitspakets
    überprüfen.
    \item \textbf{Implementation} -- Das Arbeitspaket befindet sich in der
    Implementierungsphase. Hierbei wird Code geschrieben der den Anforderungen
    des zuvor geschriebenen Tests gerecht wird.
    \item \textbf{Verifizierung} -- Das Arbeitspaket wurde fertiggestellt und
    muss nun von der Abteilung Datenverarbeitung abgesegnet werden.
    \item \textbf{Erledigt} -- Arbeitspakete in dieser Spalte sind fertiggestellt
    und vom Auftraggeber überprüft und unterschrieben worden.
\end{itemize}

Das Augenmerk ist hierbei vor Allem auf die Spalte "Verifizierung" zu legen.
Diese bereichert den Entwicklungsprozess um einen formellen, fest vorgesehenen
Zeitpunkt im Entwicklungsprozess, die dem Kunden ermöglicht, direktes Feedback
zur Lösung des Arbeitpakets zu geben, und frühzeitig möglicherweise notwendige
Anpassungen zu kommunizieren. Wichtig ist hierbei, dass die Überprüfung nicht in
einem Meeting geschieht, da sonst der Entwicklungsprozess gestört wird. Feedback
zu den einzelnen Arbeitspaketen geschieht ausschließlich über das Kanbanboard.\par


\section{Projektplanung}
\subsection{Projektphasen}
Dem Autor standen 70 Arbeitsstunden zur Umsetzung des Projektes zur Verfügung.
Vor Beginn des Projekts wurden diese auf die, für die Softwareentwicklung
typischen, Phasen verteilt. Eine grobe Übersicht ist in der unteren Tabelle zu
entnehmen. Eine ausführlichere Aufteilung der Stunden kann dem Anhang
detaillierte Zeitplanung entnommen werden.

\begin{tabular}{ |l r r r|  }
\hline
Analysephase & & & 9h \\
\hline
1. Analyse des Ist-Zustands & & 3h & \\
1.1 Fachgespräch mit der Datenverarbetungsabteilung & 1h & &  \\
1.2 Prozessanalyse & 2h & &  \\
2. "Make or Buy" Entscheidung und Wirtschaftlichkeitsanalyse & & 1h & \\
3. Erstellen eines "Use-Case" Diagramms & & 2h & \\
4. Erstellen eines Lastenhefts mit der Abteilung Datenverarbeitung & & 3h &  \\
\hline
Entwurfsphase & & & 26h \\
\hline
1. Prozessentwurf & & 2h & \\
2. Algorithmusentwurf & & 14h & \\
3. Entwurf von C Datenstrukturen & & 5h & \\
3.1 Speicherarme Q-Gramm Liste & 3h &  & \\
3.2 Selbst-ausgleichender Binär Baum & 2h & & \\
4. Entwurf der Kommandozeilenoberfläche & & 2h & \\
5. Erstellen eines UML Komponentendiagramms der Anwendung & & 4h & \\
6. Erstellen des Pflichtenhefts & & 4h & \\
\hline
Implementierungsphase & & & 21h \\
\hline
1. Unittests erstellen & & 5h & \\
2. C Datenstrukturen Implementieren & & 8h & \\
3. Algorithmus Logik in C Implementieren & & 8h & \\
\hline
Abnahmephase & & & 2h \\
\hline
1. Code-Flight mit Abteilungsleitung  & & 1h & \\
2. Präsentation & & 1h & \\
\hline
Dokumentationsphase & & & 9h \\
\hline
1. Erstellen der Nutzerdokumentation als Man Page & & 2h & \\
2. Erstellen der Projektdokumentation & & 7h & \\
\hline
Pufferzeit & & & 2h \\
\hline
1. Puffer & & 3h & \\
\hline
\hline
Gesamt & & & 70h \\
\hline
\end{tabular}


\subsection{Abweichungen vom Projektantrag}
Ursprünglich wurde im Projektantrag eine Implementierung des Algorithmus in
Python genannt. Diese wurde, nach gründlicherer Recherche und in Absprache mit der
Abteilung Datenverarbeitung jedoch auf einen späteren Zeitpunkt verlegt. Grund
hierfür ist, dass die C Implementierung höhere Priorität als die Python Einbindung hat.
Deshalb wurde beschlossen, die bisherigen 7 Stunden, welche der Python Implementierung
zugewiesen worden waren, stattdessen der Optimierung des C Programms zugeschrieben.

\subsection{Ressourcenplanung}
In der Ressourcenplanung finden sich sämtliche Einheiten aufgelistet, welche für
 die Realisierung des Projektes benötigt wurden. Hierbei ist zu erwähnen dass
 hiermit Hard- und Softwareressourcen, sowie personelle Ressourcen gemeint sind.
 Die Auflistung der Ressourcen kann im Anhang eingesehen werden.

\textbf{Hardware}
\begin{itemize}
    \item Büroarbeitsplatz mit vernetztem Rechner
    \item Laptop
\end{itemize}

\textbf{Software}
\begin{itemize}
    \item openSuSE 13.1 - Linux Distribution
    \item Atom IDE - Modulare, Open-Source Entwicklungsumgebung für eine Vielfalt an Sprachen
    \item PyCharm Community Edition - Python Entwicklungsumgebung
    \item LaTeX - Textsatzsystem zur Erstellung von Dokumenten
    \item Python 3.5 - Interpretierte Programmiersprache
    \item Cython - Superset der Python Sprache zur Einbindung von C Code.
    \item GNU Compiler Collection (GCC) - Compilersammlung des GNU Projekts
    \item Git - Dezentralisierte Versionsverwaltung
\end{itemize}

\textbf{Personal}
\begin{itemize}
    \item Mitarbeiter der Abteilung Datenverarbeitung - Festlegung der Anforderungen sowie Abnahme des Projektes
    \item Entwickler - Umsetzung des Projekts
    \item Anwendungsentwickler - Code Review
\end{itemize}



\subsection{Entwicklungsprozess}
Für die Umsetzung des Projekts wurde vom Autor ein agiler Entwicklungsprozess in
Form von Kanban entschieden. Ausschlaggebend waren hierfür die Tatsache dass es
keine festen Phasenlängen gibt, und Prioritäten sofort angepasst werden, wenn
neue Informationen(z.B. Abschluss von Arbeitspaketen) hinzukommen.\par

Das Kanbanboard wurde hierbei in 5 Bahnen aufgeteilt, welche repräsentativ für
die 5 Stufen, in welche der Autor die Entwicklungsphase unterteilt hat, stehen:
\begin{itemize}
    \item \textbf{Offen} -- Arbeitspakete in dieser Spalte befinden sich noch
    nicht in der Entwicklung, und warten darauf begonnen zu werden.
    \item \textbf{Testentwicklung} -- Es werden Tests geschrieben, welches alle
    Anforderungen an den erforderlichen Code zur Fertigstellung des Arbeitspakets
    überprüfen.
    \item \textbf{Implementation} -- Das Arbeitspaket befindet sich in der
    Implementierungsphase. Hierbei wird Code geschrieben der den Anforderungen
    des zuvor geschriebenen Tests gerecht wird.
    \item \textbf{Verifizierung} -- Das Arbeitspaket wurde fertiggestellt und
    muss nun von der Abteilung Datenverarbeitung abgesegnet werden.
    \item \textbf{Erledigt} -- Arbeitspakete in dieser Spalte sind fertiggestellt
    und vom Auftraggeber überprüft und unterschrieben worden.
\end{itemize}

Ein Augenmerk ist hierbei vor Allem auf die Spalte "Verifizierung" zu legen.
Diese bereichert den Entwicklungsprozess um einen formellen, fest vorgesehenen
Zeitpunkt im Entwicklungsprozess, die dem Kunden ermöglicht, direktes Feedback
zur Lösung des Arbeitpakets zu geben, und frühzeitig möglicherweise notwendigen
Anpassungen zu kommunizieren.\par


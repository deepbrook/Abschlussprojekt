\section{Implementierungsphase}

\subsection{Entwicklungsvorbereitung}
Zunächst wurden einige organisatorische Bedingungen erfüllt, bevor die eigentliche
Entwicklungsphase offiziell begann. So wurden das Kanban Board mit den
Aufgabenpaketen aus dem Pflichtenheft populiert. Hierzu wurden Tests und deren
dazugehörigen Funktionen separat betrachtet und als Arbeitspaket beschrieben.

\subsection{Implementierung der Datenstruktur}
Wie bereits in Sektion X erwähnt, werden zwei spezielle Datenstrukturen für das
Programm benötigt - Hashmaps und dynamisch wachsende Arrays.

\subsubsection{Hashmap Implementation}


Auf die eigene Implementierung einer Hashmap wurde verzichtet, stattdessen wurde
eine frei-verfügbare Bibliothek genutzt, welche eine Hashmap für
String-Integer Paare zur Verfügung.

Der Code wurde weitesgehend verbatim übernommen, mit der Ausnahme des
Containertyps, welches die Hashmap speichert. Da ein QGram in mehreren Strings
vorkommen kann, musste der Containertyp von einem Integer zu einer dynamisch
wachsenden Array abgewandelt werden. Da letztere ohnehin entwickelt werden muss,
war die Komplexität dieser Anforderung gering.

\subsubsection{Dynamische Array Implementation}
C besitzt von Haus heraus nur simple Datentypen, welche dem Nutzer erlauben
exponentiell komplexe Datenstrukturen zu erschaffen. Da auch eine dynamische wachsende
Array nicht in ANSI-C zur Verfügung steht, musste eine Datenstruktur für diese erstellt werden.
Das Konzept hierbei ist recht einfach - sollte die Array voll sein, so wird beim
nächsten Aufruf der Insert Funktion die Länge der Array verdoppelt. Hierfür wurde eine
neue Datenstruktur angelegt und 3 Funktionen definiert, um mit der Datenstruktur
zu interagieren. Ein Auszug des Quellcodes dieser Datenstruktur is auf Seite X
im Anhang vorzufinden.\par

\subsection{Implementierung der Benutzeroberfläche}

\clearpage
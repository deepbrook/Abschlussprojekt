\section{Implementierungsphase}
\label{section:implementierungsphase}
\subsection{Entwicklungsvorbereitung}
Zunächst mussten einige organisatorische Bedingungen erfüllt werden, bevor die eigentliche Entwicklungsphase offiziell beginnen konnte. So wurden das Kanban Board mit den
Aufgabenpaketen aus dem Pflichtenheft populiert. Hierzu wurden Tests und deren
dazugehörige Funktionen betrachtet und als Arbeitspaket beschrieben.

\subsection{Implementierung der Datenstrukturen}
Wie bereits im vorherigen Abschnitt erwähnt, werden zwei spezielle Datenstrukturen für das Programm benötigt -- Hashmaps und dynamisch wachsende Arrays.

\subsubsection{Implementierung der Hashmap}


Auf die eigene Implementierung einer Hashmap wurde verzichtet, stattdessen wurde
eine frei verfügbare Bibliothek genutzt, welche eine Hashmap für
String-Integer Paare zur Verfügung stellt.

Der Code wurde weitgehend verbatim übernommen, mit der Ausnahme des
Containertyps, welchen die Hashmap speichert. Da ein Q-Gramm in mehreren Strings
vorkommen kann, musste der Containertyp von einem Integer zu einem dynamisch
wachsenden Array abgewandelt werden. Da letzteres ohnehin entwickelt werden muss,
war die Komplexität dieser Anforderung gering.

\subsubsection{Implementierung des Dynamisch wachsenden Arrays}
C stellt nur simple Datentypen zur Verfügung, welche dem Nutzer erlauben komplexere
Datenstrukturen abzubilden. Da ein dynamisch-wachsendes
Array nicht in C zur Verfügung steht, musste eine Datenstruktur für dieses erstellt werden.
Das Konzept hierbei ist recht einfach: Sollte das Array voll sein, so wird beim
nächsten Aufruf der Insert-Funktion die Länge des Array verdoppelt. Hierfür wurde
eine neue Datenstruktur angelegt und drei Funktionen definiert, um mit dem Array
interagieren zu können.\par

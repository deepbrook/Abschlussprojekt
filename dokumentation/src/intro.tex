\section{Einleitung}
\subsection{Projektumfeld}
Die GA Financial Solutions GmbH ist ein fünfköpfiger Finanzdienstleister im
Herzen Berlins, welcher sich auf computergestütztes Handeln
von Futures, anderen Derivate und Aktien spezialisiert hat. Direkte Kunden gibt es zur Zeit
nicht und die Firma finanziert sich durch Investoren und den Gewinn ihrer
Strategien. Die Rolle des Auftraggebers und Kunden übernimmt im Rahmen des
Projektes Sebastian Freundt, stellvertretend für die Abteilung Datenverarbeitung.\par

\subsection{Projektziel}
Ziel ist es, ein Kommandozeilenprogramm zu schreiben, welches mehrere Derivatsymbole 
von verschiedenen Anbietern anhand einer Metrik vergleicht und übereinstimmende 
Strings ausgibt.\par

\subsection{Projektbegründung}
Die GA Financial Solutions GmbH handelt Wertpapiere aller Art auf eigene
Rechnung.  Dabei ist es marktüblich, dass Referenzdaten, Preisdaten
und Ausführungen von unterschiedlichen Dienstleistern bezogen werden.
So stehen ca. 223 Mio. handelbaren Wertpapieren rund 500.000 mögliche
Herausgeber an ca. 1.200 Börsen weltweit gegenüber.  Jede einzelne
Börse bzw. jeder einzelne Makler im Markt unterstützt jedoch nur
einen Bruchteil der Papiere.\par

In Ermangelung eines weltweiten Standards, der jedem Marktteilnehmer in
jeder Handelsphase gerecht wird, werden unterschiedliche Symbologien
benutzt, deren kleinster gemeinsamer Nenner stets die
Klartextbezeichnung des gegebenen Papiers zusammen mit der
Klartextbezeichnung des Herausgebers ist.  Erschwert wird die
Problematik überdies noch durch unterschiedliche Transliterationsverfahren,
so führt z.B. die NASDAQ die Firma "`Panasonic Corp."',
die GLEIS-Datenbank\footnote{Global Legal Entity Identifier System} jedoch unter
\begin{CJK}{UTF8}{min}
"パナソニック株式会社"
\end{CJK}%
, was transskribiert wiederum "`Panasonikku Kabushiki-gaisha"' ergibt. \par

Bei der GA Financial Solutions GmbH werden Zuordnungen zwischen Symbologien
aktuell ad hoc mit Heuristiken und in Handarbeit bewerkstelligt.  Der
Zeitaufwand ist entsprechend hoch, die Fehlerrate ebenfalls.  Es
verbietet sich geradezu, die bisherigen Heuristiken systematisch zur
Zuordnung aller Herausgeber aller Wertpapiere zu benutzen.\par

Das zu entwickelnde Tool soll vor allem den Zeitaufwand und die benötigte
menschliche Komponente, und somit Fehlerquellen, reduzieren, so dass diese Ressourcen entsprechend
anderen Aufgaben zugeteilt werden können.\par

\subsection{Projektschnittstellen}
Das Programm wird als Kommandozeilenprogramm angeboten, welches
Daten per Pipe oder unter Angabe eines Dateipfads annimmt und diese im .csv 
Format über STDOUT ausgibt. Dieser Ansatz der Resultatausgabe wird explizit gewünscht, 
da sich das Programm auf diese Art mühelos in die bisherige Toolchain der Firma einbinden lässt.
Das Programm wird vor allem in der Abteilung für Datenverarbeitung zum Einsatz kommen.\par

Als technische Schnittstelle ist insbesondere das Datensammelsystem zu nennen,
welches täglich unsere Datenbanken mit neuen Datensätzen füttert. Die Ergebnisse
dieses Systems liegen im CSV-Format auf unseren Servern, bevor sie am Ende des
Tages komprimiert werden. Eine direkte Anbindung des zu entwickelnden Projekts an das Datensammelsystem findet nicht statt. Wichtig ist allerdings, dass das Programm CSV-Dateien unterstützt, welche vom System bereitgestellt werden.\par

Mittel und Genehmigung des Projekts stellt die Abteilung Datenverarbeitung,
vertreten durch Sebastian Freundt.\par

\subsection{Projektabgrenzung}
Während die Hauptziele des Projekts das Vergleichen mehrerer hunderttausend Symbole und die effiziente Gestaltung dieser Vergleiche sind, sind folgende Themen explizit nicht Teil der Aufgabe:

\begin{itemize}
    \item Datenintegrität -- Die Überprüfung des Datenformats ist nicht Teil des Aufgabenfeldes.
    \item Datenverifizierung -- Die Prüfung, ob \textbf{eingehende} Daten korrekt sind, fällt nicht in das Aufgabenfeld.
    \item Datenbereitstellung -- Die Bereitstellung der verarbeiteten Daten (z.B. als Datenbank o.Ä.) ist ebenfalls nicht Teil des Projekts.
\end{itemize}
\clearpage


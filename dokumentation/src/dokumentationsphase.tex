\section{Dokumentation}
Die Dokumentation ist aufgeteilt in zwei Bestandteile: Die Nutzerdokumentation
in Form einer CLI Dokumenation, sowie die Entwicklerdokumentation in Form von
Dokumentationsblöcken im Quellcode. Die gesamte Dokumentation wurde auf Englisch,
der Sprache des Unternehmens, verfasst.

\subsection{Nutzerdokumentation}
Da der Funktionsumfang des Programms recht klein ist, wird auf ein handelsübliches
Dokument verzichtet. Die Dokumentation findet ausschließlich über die Kommandozeile, über den Aufruf des Programms mit dem Parameter "-h" bzw. "--help", statt.
Dieser Aufrauf erzeugt eine kurze Beschreibung des Programms und dessen Parameter.

\subsection{Entwicklerdokumentation}
Die Entwicklerdokumentation ist etwas weitreichender gestaltet, jedoch bewusst minimalistisch gehalten. Die Dokumentation des Codes findet über einen
Dokumentationsblock über der Signatur jeder Funktion statt. Dieser Block beinhaltet
eine kurze Beschreibung, sowie die Namen und erwartete Datentypen der Funktionsparameter.

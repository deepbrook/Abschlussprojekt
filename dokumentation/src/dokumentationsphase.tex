\section{Dokumentation}
\label{section:dokumentationsphase}
Die Dokumentation ist aufgeteilt in zwei Bestandteile: Die Nutzerdokumentation
in Form einer CLI-Dokumentation, sowie die Entwicklerdokumentation in Form von
Dokumentationsblöcken im Quellcode. Die gesamte Dokumentation wurde auf Englisch,
der Sprache des Unternehmens, verfasst.

\subsection{Nutzerdokumentation}
Da der Funktionsumfang des Programms recht klein ist, wird auf ein handelsübliches
Dokument verzichtet. Die Dokumentation findet ausschließlich über die Kommandozeile mit dem Aufruf des Programms unter Zusatz des Hilfe-Parameters (''help'' bzw. ''h'') statt.
Dieser Aufruf erzeugt eine kurze Beschreibung des Programms und dessen Parameter.

\subsection{Entwicklerdokumentation}
Die Entwicklerdokumentation des Codes findet über einen
Dokumentationsblock oberhalb der Signatur jeder Funktion statt. Dieser Block
beinhaltet eine kurze Beschreibung der Funktionsweise der Routine sowie die
Namen und erwarteten Datentypen der Funktionsparameter. Die Formatierung ist
ReStructuredText (kurz ReST), kann mit Hilfe von Dokumentationstools ausgelesen
und später als eine lokale Webseite mit automatisch generierten Verlinkungen ausgegeben werden.

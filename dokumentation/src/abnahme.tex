\section{Abnahmephase}
\label{section:abnahmephase}
Nach Abschluss der Implementierungsphase wurde das Projekt, wie zuvor mit der
Abteilung für Datenverarbeitung abgesprochen, sowohl dem Abteilungsleiter im Einzelgespräch,
als auch der gesamten Abteilung in Form einer Präsentation, präsentiert.

Das Hauptaugenmerk und die Vorgehensweise war in beiden Fällen jedoch unterschiedlich
und wird in den folgenden Aschnitten erläutert.

\subsection{Code-Flight mit der Abteilungsleitung}

Um die Qualität des Codes zu gewährleisten wurde für das Projekt die Abnahme über einen Code-Flight\footnote{zu deutsch \"Programm-Flug"} festgelegt. Diese einstündige
Vorführung des Codes wurde mit dem Abteilungsleiter Sebastian Freundt im
Vier-Augen-Gespräch vollzogen und gestaltete sich wie folgt:

\begin{itemize}
    \item Demonstration des Programms über die Kommandozeile, mit Schritt-für-Schritt Erklärung der gewählten Parameter und deren Design-Entscheidung.
    \item Sichtung des Codes, angefangen mit der Hauptroutine.
    \item Erläuterung der gewählten Datenstrukturen und Begründung für die Nutzung dieser.
    \item Limitation des Algorithmus und des CLI-Programms.
    \item Weitere Schritte zur Verbesserung der Hauptprogramms.
    \item Mögliche Ansätze zur Verbesserung des Algorithmus.
\end{itemize}

\subsection{Präsentation des Programms}

Nachdem Code-Flight mit dem Abteilungsleiter, wurde ebenfalls
eine Präsentation für das gesamte Team der Abteilung Datenverarbeitung gehalten.
Diese beschränkte sich im Inhalt jedoch auf die Grundprinzipien des Algorithmus, sowie
die Funktion und Nutzung des CLI-Programms. Unter anderem wurden verfügbare
Parameter und maximal zulässige Dateigrößen erläutert.

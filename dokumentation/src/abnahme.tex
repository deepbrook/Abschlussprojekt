\section{Abnahmephase}
\label{section:abnahmephase}
Nach Abschluss der Implementierungsphase wurde das Projekt, wie zuvor mit der
Abteilung für Datenverarbeitung abgesprochen, sowohl dem Abteilungsleiter im Einzelgespräch
als auch der gesamten Abteilung in Form einer Präsentation vorgestellt.

Das Hauptaugenmerk und die Vorgehensweise war in beiden Fällen jedoch unterschiedlich
und wird in den folgenden Abschnitten erläutert.

\subsection{Code-Flight mit der Abteilungsleitung}

Um die Qualität des Codes zu gewährleisten, wurde für das Projekt die Abnahme über einen Code-Flight\footnote{zu deutsch "`Programm-Flug"´ - bei dieser Technik erklärt eine Person dem Zuhörer laut den Code und beschreibt was dessen Funktion ist.} festgelegt. Diese einstündige
Vorführung des Codes wurde mit dem Abteilungsleiter Sebastian Freundt im
Vier-Augen-Gespräch vollzogen und gestaltete sich wie folgt:

\begin{itemize}
    \item Demonstration des Programms über die Kommandozeile, mit Schritt-für-Schritt-Erklärung der gewählten Parameter und deren Design-Entscheidung
    \item Sichtung des Codes, angefangen mit der Hauptroutine
    \item Erläuterung der gewählten Datenstrukturen und Begründung für die Nutzung dieser
    \item Limitation des Algorithmus und des CLI-Programms
    \item Weitere Schritte zur Verbesserung der Hauptprogramms
    \item Mögliche Ansätze zur Verbesserung des Algorithmus
\end{itemize}

\subsection{Präsentation des Programms}

Nach dem Code-Flight mit dem Abteilungsleiter wurde für das gesamte Team der Abteilung Datenverarbeitung ebenfalls eine Präsentation gehalten.
Diese beschränkte sich im Inhalt jedoch auf die Grundprinzipien des Algorithmus sowie
die Funktion und Nutzung des CLI-Programms. Unter anderem wurden verfügbare
Parameter und maximal zulässige Dateigrößen erläutert.
Abschließend wurde ein Post Mortem gehalten und die gewonnenen Erkenntnisse in Bezug auf die Arbeit mit Kanban und der testgetriebenen Entwicklung erläutert.
\section{Fazit}

\subsection{Soll- /Ist-Vergleich}
Rückblickend kann festgehalten werden, dass alle, im Pflichtenheft
abgesprochenen Funktionen und Anforderungen eingehalten worden sind,
und zur vollsten Zufriedenheit des Auftraggebers erstellt wurden. Hierbei
spielt die Entscheidung, die Einbindung des C Codes in PYthon zu streichen, sicherlich
eine elementare Rolle. Wie man erkennen kann, wurden die Stunden, welche durch
die Streichung frei wurden, effektive und vollsten ausgenutzt um den Anforderungen
des Projekts gerecht zu werden. Dabei wurden die von der IHK vorgegebenen 70 Stunden
konsequent eingehalten.

\subsection{Post Mortem}

Vor Allem die Entwurfsphase half dem Autor seine Kenntnisse aus dem Projektmanagement anzuwenden,
und waren bei den Nutzwertanalysen besonders hilfreich. Auch stellte sich die Anwendung des
testgetriebenen Entwicklungskreislaufs als große Unterstützung dar - so wurden lästige Segmentation Faults frühzeitig
erkannt, und die Funktionen (und somit das gesamt Programm) konnten schneller zu Produktionsreife heranwachsen.
Auch das genutzte Kanban-Board erwies sich als großartiges Werkzeug, um Prozesse und deren Fortschritt zu dokumentieren.
Kurze Kommunikations- und Entshceidungswege ermöglichten eine agile Entwicklung der Projekts und werden
dem Autor auch weiterhin als ein wichtiger Baustein produktiver Arbeitsumgebungen in Erinnerung bleiben.

\subsection{Ausblick}
Punkte, welche im Lastenheft genannt, aber nicht mit in das Pflichtenheft genommen worden,
stehen nun natürlich auf der Liste der weiteren Schritte. Da der Code gut dokumentiert
und mit viel Auge für Lesbarkeit geschrieben worden ist, ist eine Cythoneinbindung nun trivial.
Durch die bereitgestellten Tests können ebenfalls schnell Python Unittests erstellt, und somit
die Einbindung des C Codes effektiv und verlässlich auch in Python überprüft werden.

Schlussendlich bleiben noch zwei Punkte, welche in Zukunft angegangen werden könnten, um das Programm weiter zu verfeinern:
\begin{enumerate}

\item Die Optimierung der Programmlogik \\
    \begin{itemize}
        \item Verfeinerung der Hashfunktion -
        \item
    \end{itemize}


\item Die Anpassung des Q-Gramm Filters
    \begin{itemize}
        \item
        \item
    \end{itemize}

\end{enumerate}



